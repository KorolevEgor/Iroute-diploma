В современном мире навигационные приложения стали неотъемлемой частью повседневной жизни, облегчая перемещение и ориентацию в городской среде. Они помогают пользователям находить оптимальные маршруты для достижения разнообразных целей: поездок на работу, прогулок по городу, поиска интересных мест и даже спортивных тренировок. Несмотря на широкий спектр возможностей, предоставляемых навигационными приложениями, большинство из них уделяют основное внимание минимизации времени, что не всегда соответствует ожиданиям пользователей.

Сегодняшний активный образ жизни требует от нас более гибкого подхода к навигации. Многие люди стремятся не только эффективно перемещаться по городу, но и контролировать свою физическую активность и поддерживать ее на определенном уровне. Подход, рассматриваемый в данной работе, позволяет строить маршруты не только с целью достижения места назначения, но и построить его с учетом желаемой дистанции, что особенно актуально для занятий спортом, прогулок и здорового образа жизни.

В данной работе представляется новый подход к построению маршрутов на картах, который уделяет внимание заданной дистанции, которую пользователь желает преодолеть. Описанный алгоритм направленного блуждания на графе позволяет строить маршруты для активных прогулок, бега или велосипедных поездок.

Целью данной работы является разработка и описание алгоритма построения маршрутов с учетом заданной дистанции, а также демонстрация его применения в навигационном веб-приложении. В работе предоставляется подробное описание методологии алгоритма и его реализации, а также демонстрируется работа разработанного приложения. Данное приложение может быть востребовано среди пользователей, занимающихся циклическими видами спорта, а также среди людей, которые стремятся контролировать свою физическую активность.
