\iffalse
Текст текст текст текст текст\\

\section{Название подраздела 1}

\begin{figure}[H]
	\begin{center}
		\includegraphics[width=0.7\linewidth]{src/img/img_example.png}
		\caption{Пример вставки изображения}
		\label{fig:img_example}
	\end{center}
\end{figure}

\section{Название подраздела 2}

\begin{center}
	%\resizebox{\textwidth}{!}{
		\begin{tabular}{|c|c|c|}
			\hline
			A1 & B1 & C1 \\ \hline
			A2 & B2 & C2 \\ \hline
		\end{tabular}
		%}
\end{center}
\captionof{table}{Пример вставки таблицы}
\label{tab:tab_example}
\fi

\section{Теория графов}
Теория графов представляет собой обширную область математики, которая изучает структуры, состоящие из вершин и рёбер, а также их свойства. В контексте разработки рекомендательной системы для генерации маршрутов, где необходимо учитывать фиксированную дистанцию, а также пользовательские фильтры, понимание основных концепций теории графов имеет важное значение. Данный раздел представляет обзор ключевых аспектов теории графов и их применение.

Истоки теории графов уходят в XVIII век, когда математики начали исследовать проблемы, связанные с сетями, коммуникациями и транспортными системами. Одним из первых, кто активно развивал теорию графов, был Леонард Эйлер. В 1736 году он опубликовал статью, в которой предложил общий метод решения задач о прохождении мостов, представив карту Кёнигсберга в виде графа. Это и стало началом развития теории графов как самостоятельной математической дисциплины.

%В дальнейшем теория графов развивалась благодаря работам таких учёных, как Аугустин Кайршль, Петер Гюйгенс, Уильям Роуан Хэмилтон и других. Важным вехом стала работа Артура Кэли и Фрэнсиса Киркмана, которые в 1857 году опубликовали теорему о связности графов.
%
%В XX веке теория графов получила новый импульс развития благодаря работам таких учёных, как Клод Берж, Джордж Полиа, Дейвид Кёниг, Пол Эрдёш, Джон Хопкрофт и Роберт Тарьян, которые внесли значительный вклад в область алгоритмов, комбинаторики и приложений теории графов.

С появлением компьютеров и развитием информационных технологий теория графов стала не только важным математическим инструментом, но и активно применяется в практических областях, таких как компьютерные сети, социальные сети, биоинформатика, логистика и другие. Её методы и подходы стали неотъемлемой частью современной науки и техники.

\subsection{Задача о кёнигсбергских мостах}

d

\subsection{Задача о четырёх красках}

d

\subsection{Основные объекты теории графов}

\begin{itemize}
	\item графом G называется пара $(V,E)$, где $V$ - множество вершин (узлов), а $E$ - множество рёбер (связей) между этими вершинами.
	\item Ориентированный и неориентированный графы: В неориентированном графе рёбра не имеют направления, в то время как в ориентированном каждое ребро имеет направление от одной вершины к другой.
	\item Связность: Граф называется связным, если между любыми двумя его вершинами существует путь.
	\item Степень вершины: Для неориентированных графов степень вершины - это количество рёбер, связанных с данной вершиной. В ориентированных графах учитываются входящие и исходящие рёбра.
	\item Подграф: Подграф графа G - это граф, вершины и рёбра которого являются подмножествами вершин и рёбер графа G.
	\item Деревья: Дерево - это связный граф без циклов. Каждая вершина в дереве имеет ровно одну входящую связь, за исключением корневой вершины, которая не имеет входящих связей.
	\item Пути и циклы: Путь - это последовательность вершин, в которой каждая пара соседних вершин соединена ребром. Цикл - это путь, в котором начальная и конечная вершины совпадают.
	\item Матрица смежности и список смежности: Матрица смежности - это квадратная матрица, где элемент $a_{ij}$ равен $1$, если между вершинами $i$ и $j$ есть ребро, и $0$ в противном случае. Список смежности представляет граф в виде списка, где каждая вершина сопоставляется со списком вершин, с которыми она связана.
\end{itemize}

Эти основные концепции теории графов обеспечивают базовый фреймворк для анализа и решения различных задач, связанных с графами. В дальнейшем в диссертации мы будем исследовать более сложные алгоритмы и приложения, использующие графовые структуры.


\subsection{Основные теоремы и их следствия}

\begin{itemize}
	\item Теорема о связности графа:
	
	Результат: Граф является связным тогда и только тогда, когда между любыми двумя его вершинами существует путь.
	
	Применение: Определение связности играет ключевую роль в различных задачах, таких как сетевой маршрутизации, поиск компонент связности в графах и многих других.
	
	\item Теорема Эйлера о планарных графах:
	
	Результат: Планарный граф может быть нарисован на плоскости так, чтобы его рёбра не пересекались.
	
	Применение: Эта теорема полезна в проектировании схем схемотехники, дорожных сетей и других структур, где важно избегать пересечений.
	
	\item Теорема о минимальном остовном дереве (МОД):
	
	Результат: Для связного неориентированного графа с весами на рёбрах существует уникальный МОД, содержащий все вершины графа и имеющий минимальную сумму весов рёбер.
	
	Применение: Применяется в задачах оптимизации, таких как минимальное остовное дерево в сетях связи и транспортных сетях.
	
	\item Теорема о потоках и разрезах (теорема Форда-Фалкерсона):
	
	Результат: Для любого потока в графе максимальный поток равен минимальному разрезу.
	
	Применение: Используется для решения задач максимального потока и минимального разреза в сетях, таких как транспортные сети и сети электропередачи.
	
	\item Теорема о четности степеней вершин в неориентированных графах:
	
	Результат: В неориентированном графе количество вершин нечетной степени всегда четно.
	
	Применение: Эта теорема используется в различных задачах, включая проверку наличия эйлерова цикла в графе.
	
	\item Теорема о цветовой раскраске графа (теорема о четырёх красках):
	
	Результат: Любой плоский граф может быть правильно раскрашен с использованием не более четырёх различных цветов.
	
	Применение: Применяется в картографии для раскрашивания карт так, чтобы соседние регионы имели разные цвета, и в различных задачах графического моделирования, требующих минимизации числа используемых цветов.
	
	\item Теорема Кёнига о покрытиях в двудольных графах:
	
	Результат: В каждом двудольном графе количество рёбер в минимальном покрытии равно числу вершин в максимальном паросочетании.
	
	Применение: Используется в задачах оптимизации, например, в различных алгоритмах для планирования расписания, а также в теории сетей для оптимизации распределения ресурсов.
	
	\item Теорема Холла о паросочетаниях:
	
	Результат: Для любого двудольного графа G с долями X и Y существует паросочетание, покрывающее все вершины X, если и только если для любого подмножества вершин S из X мощность множества соседей $N(S)$ больше или равна мощности S.
	
	Применение: Используется в задачах, связанных с назначением ресурсов или соединений в сетях, таких как задачи назначения ресурсов и соединений в сетях транспортировки или коммуникаций.
	
	\item Теорема Меньгера:
	
	Результат: В неориентированном графе минимальное количество вершин, разделяющих две заданные вершины s и t, равно максимальному количеству непересекающихся путей между s и t.
	
	Применение: Используется в задачах нахождения наиболее эффективных маршрутов, например, в транспортной логистике или сетях связи.
	
	\item Теорема Дирака о гамильтоновых циклах:
	
	Результат: Если в графе G с n вершинами каждая вершина имеет степень не менее n/2, то граф содержит гамильтонов цикл.
	
	Применение: Применяется в задачах, требующих нахождения замкнутых маршрутов, таких как в области транспортировки или проектировании эффективных обходов для механизмов.
	
	\item Теорема Мостов Кёнига о деревьях:
	
	Результат: В любом связном графе с $n$ вершинами $n-1$ ребро является достаточным и необходимым условием для того, чтобы он был деревом.
	
	Применение: Эта теорема используется для проверки наличия циклов в графах и в задачах поиска минимальных остовных деревьев.
	
	\item Теорема Штайнера:
	
	Результат: Для заданных точек на плоскости (называемых узлами) и некоторого числа дополнительных точек (называемых точками Штайнера) существует граф, содержащий только данные точки и имеющий минимальную длину.
	
	Применение: Эта теорема используется в различных задачах, таких как проектирование схем коммуникаций или сетей, где нужно минимизировать длину кабелей или стоимость связи.
	
	\item Теорема Турана о толстых подграфах:
	
	Результат: Для любого графа с n вершинами и без треугольников максимальное количество рёбер ограничено сверху.
	
	Применение: Эта теорема используется в задачах, связанных с поиском плотных подграфов или определением верхней границы количества рёбер в графе.
\end{itemize}

\subsection{Приложения теории графов}

\begin{itemize}
	\item Социальные сети: Социальные сети можно представить в виде графов, где узлы представляют пользователей, а рёбра - связи между ними. Анализ графа социальной сети позволяет исследовать структуру сети, выявлять влиятельных пользователей, группы схожих интересов и т. д.
	
	\item Маршрутизация в компьютерных сетях: Графы используются для моделирования компьютерных сетей, где узлы представляют маршрутизаторы или узлы сети, а рёбра - связи между ними. Различные алгоритмы маршрутизации, такие как алгоритм Дейкстры или алгоритм Беллмана-Форда, используются для нахождения оптимальных путей между узлами сети.
	
	\item Биоинформатика: Графы используются для моделирования биологических сетей, таких как генетические сети, белковые взаимодействия и др. Анализ таких графов позволяет исследовать сложные взаимодействия в биологических системах и выявлять ключевые элементы.
	
	\item Маршрутизация и логистика: В транспортной логистике графы используются для моделирования сетей дорог, железных дорог, авиалиний и т. д. Алгоритмы маршрутизации помогают оптимизировать распределение грузов и пассажирские перевозки, минимизируя время и затраты.
	
	\item Финансовая аналитика: Графы могут использоваться для моделирования финансовых сетей, таких как сети финансовых учреждений, банковских транзакций или связей между компаниями. Анализ таких сетей позволяет выявлять системные риски, связанные с финансовыми кризисами и колебаниями на рынке.
	
	\item Графовые базы данных: Графовые базы данных используются для хранения и анализа связанных данных, таких как социальные сети, транспортные сети, биологические сети и др. Они предоставляют эффективные методы для выполнения запросов и анализа связанных данных.
	
	\item Оптимизация процессов: Графы используются для моделирования процессов в производстве, логистике, телекоммуникациях и других отраслях. Анализ графа позволяет оптимизировать производственные процессы, управлять инвентаризацией, распределением ресурсов и др.
	
	\item Робототехника и навигация: Графы используются для моделирования окружающей среды роботов и планирования их движения. Алгоритмы поиска пути и управления роботами основаны на анализе графов для эффективного перемещения в пространстве.
\end{itemize}


\subsection{Алгоритмы на графах}

Рассмотрим несколько ключевых алгоритмов на графах, которые широко используются в различных областях:

\begin{itemize}
	\item Алгоритм поиска в ширину (BFS):

	Описание: BFS исследует граф от заданной стартовой вершины, поочередно обходя все ближайшие к ней вершины, затем переходя к следующему уровню.
	Применение: Используется для поиска кратчайшего пути в невзвешенном графе, определения связности графа, поиска кратчайшего пути в графе, представленном в виде дерева, и др.
	
	\item Алгоритм поиска в глубину (DFS):
	
	Описание: DFS исследует граф до тех пор, пока не достигнет конечной вершины, а затем возвращаетсть назад и продолжает поиск.
	Применение: Используется для нахождения компонент связности в графе, топологической сортировки, поиска циклов в графе, проверки наличия пути между вершинами и др.
	
	\item Алгоритм Дейкстры:
	
	Описание: Алгоритм Дейкстры находит кратчайший путь от одной вершины графа до всех остальных, при условии, что веса рёбер неотрицательны.
	Применение: Используется для поиска кратчайших путей в графах с весами, например, в сетях передачи данных, транспортных сетях и т. д.
	
	\item Алгоритм Беллмана-Форда:
	
	Описание: Алгоритм Беллмана-Форда находит кратчайшие пути от одной вершины графа до всех остальных, даже при наличии рёбер с отрицательным весом, но не содержащих отрицательные циклы.
	Применение: Используется для нахождения кратчайших путей в графах с весами, где могут присутствовать рёбра с отрицательным весом, но нет отрицательных циклов.
	
	\item Алгоритм Прима:
	
	Описание: Алгоритм Прима находит минимальное остовное дерево взвешенного связного графа.
	Применение: Используется в задачах минимизации стоимости, таких как проектирование сетей передачи данных, планирование транспортных маршрутов и др.
	
	\item Алгоритм Крускала:
	
	Описание: Алгоритм Крускала также находит минимальное остовное дерево, но он работает не построением дерева, а пошаговым добавлением рёбер с минимальным весом, не образующих цикл.
\end{itemize}

Применение: Используется в задачах оптимизации, связанных с построением минимальных сетей связи, электропередачи и др.
Эти алгоритмы являются лишь некоторыми из множества методов и подходов, применяемых в теории графов для решения различных задач. Каждый из них имеет свои особенности и применяется в зависимости от требований и характеристик конкретной задачи.


\section{Вероятностные графовые модели}

d

%2.1 Пути и циклы в графах
%Пути и циклы - это основные элементы, используемые для описания путей перемещения в графах.

%Пути: Путь в графе представляет собой последовательность вершин, в которой каждая вершина соединена ребром с последующей вершиной. Путь может быть простым или составным, в зависимости от наличия повторяющихся вершин.
%Циклы: Цикл в графе - это путь, в котором начальная и конечная вершины совпадают, и ни одна вершина не повторяется, кроме начальной и конечной. Они могут быть простыми или составными, а также могут иметь различные свойства, влияющие на структуру графа и возможные маршруты.
%2.2 Изоморфизм и деревья в графах
%Изоморфизм и деревья - это концепции, которые помогают понять структуру и связи между вершинами в графах.

%Изоморфизм графов: Графы называются изоморфными, если существует биективное отображение между их множествами вершин, сохраняющее отношение смежности. Понимание изоморфизма позволяет выявлять схожие или одинаковые паттерны в различных сетевых структурах.
%Деревья: Дерево - это связный ациклический граф. Они играют важную роль в моделировании иерархических структур и имеют широкое применение в алгоритмах маршрутизации и анализе данных.
%2.3 Клики и полное упорядочивание графов
%Клики и полное упорядочивание графов представляют собой дополнительные аспекты структуры графов.

%Клики: Клика - это подграф, в котором каждая вершина соединена с каждой другой вершиной. Клики играют важную роль в анализе социальных сетей и обнаружении сообществ.
%Полное упорядочивание: Полное упорядочивание - это такое отношение порядка на множестве вершин, при котором для любых двух вершин одна из них достижима из другой. Понимание полного упорядочивания важно для анализа структуры графа и применимости различных алгоритмов.
%2.4 Алгоритмы упорядочивания и триангуляции графов
%Алгоритмы упорядочивания и триангуляции позволяют анализировать и оптимизировать графы для эффективной работы рекомендательных систем.

%Алгоритмы упорядочивания: Существует множество алгоритмов для упорядочивания графов, таких как алгоритм Тарьяна и алгоритм Хопкрофта-Тарьяна, которые находят применение в различных областях, включая географическую информационную систему (ГИС) и компьютерную графику.
%Триангуляция графов: Триангуляция позволяет разбить граф на треугольники, что упрощает анализ и обработку данных. Она широко используется в графических системах и вычислительной геометрии.