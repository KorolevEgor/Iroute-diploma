Текст текст текст текст текст\\

\section{Название подраздела 1}

\begin{figure}[H]
	\begin{center}
		\includegraphics[width=0.7\linewidth]{src/img/img_example.png}
		\caption{Пример вставки изображения}
		\label{fig:img_example}
	\end{center}
\end{figure}

\section{Название подраздела 2}

\begin{center}
	%\resizebox{\textwidth}{!}{
		\begin{tabular}{|c|c|c|}
			\hline
			A1 & B1 & C1 \\ \hline
			A2 & B2 & C2 \\ \hline
		\end{tabular}
		%}
\end{center}
\captionof{table}{Пример вставки таблицы}
\label{tab:tab_example}


\section{Теория графов}
Теория графов представляет собой обширную область математики, которая изучает структуры, состоящие из вершин и рёбер, и их свойства. В контексте разработки рекомендательных систем для маршрутизации, где необходимо учитывать фиксированную дистанцию, а также пользовательские фильтры, понимание основных концепций теории графов имеет важное значение. Данный раздел представляет обзор ключевых аспектов теории графов и их применение в контексте дорожных сетей.

Истоки теории графов уходят в XIX век, когда математики начали исследовать проблемы, связанные с сетями, коммуникациями и транспортными системами. В то время Густав Кирхгофф и Густав Плейгель выработали идеи, которые впоследствии сыграли ключевую роль в формировании теории графов.

Одним из первых, кто активно развивал теорию графов, был Леонард Эйлер. В 1736 году он опубликовал статью, в которой предложил общий метод решения задач о прохождении мостов, представив карту Кёнигсберга в виде графа. Это и стало началом развития теории графов как самостоятельной математической дисциплины.

В дальнейшем теория графов развивалась благодаря работам таких учёных, как Аугустин Кайршль, Петер Гюйгенс, Уильям Роуан Хэмилтон и других. Важным вехом стала работа Артура Кэли и Фрэнсиса Киркмана, которые в 1857 году опубликовали теорему о связности графов.

В XX веке теория графов получила новый импульс развития благодаря работам таких учёных, как Клод Берж, Джордж Полиа, Дейвид Кёниг, Пол Эрдёш, Джон Хопкрофт и Роберт Тарьян, которые внесли значительный вклад в область алгоритмов, комбинаторики и приложений теории графов.

С появлением компьютеров и развитием информационных технологий теория графов стала не только важным математическим инструментом, но и активно применяется в практических областях, таких как компьютерные сети, социальные сети, биоинформатика, логистика и другие. Её методы и подходы стали неотъемлемой частью современной науки и техники.












2.1 Пути и циклы в графах
Пути и циклы - это основные элементы, используемые для описания путей перемещения в графах.

Пути: Путь в графе представляет собой последовательность вершин, в которой каждая вершина соединена ребром с последующей вершиной. Путь может быть простым или составным, в зависимости от наличия повторяющихся вершин.
Циклы: Цикл в графе - это путь, в котором начальная и конечная вершины совпадают, и ни одна вершина не повторяется, кроме начальной и конечной. Они могут быть простыми или составными, а также могут иметь различные свойства, влияющие на структуру графа и возможные маршруты.
2.2 Изоморфизм и деревья в графах
Изоморфизм и деревья - это концепции, которые помогают понять структуру и связи между вершинами в графах.

Изоморфизм графов: Графы называются изоморфными, если существует биективное отображение между их множествами вершин, сохраняющее отношение смежности. Понимание изоморфизма позволяет выявлять схожие или одинаковые паттерны в различных сетевых структурах.
Деревья: Дерево - это связный ациклический граф. Они играют важную роль в моделировании иерархических структур и имеют широкое применение в алгоритмах маршрутизации и анализе данных.
2.3 Клики и полное упорядочивание графов
Клики и полное упорядочивание графов представляют собой дополнительные аспекты структуры графов.

Клики: Клика - это подграф, в котором каждая вершина соединена с каждой другой вершиной. Клики играют важную роль в анализе социальных сетей и обнаружении сообществ.
Полное упорядочивание: Полное упорядочивание - это такое отношение порядка на множестве вершин, при котором для любых двух вершин одна из них достижима из другой. Понимание полного упорядочивания важно для анализа структуры графа и применимости различных алгоритмов.
2.4 Алгоритмы упорядочивания и триангуляции графов
Алгоритмы упорядочивания и триангуляции позволяют анализировать и оптимизировать графы для эффективной работы рекомендательных систем.

Алгоритмы упорядочивания: Существует множество алгоритмов для упорядочивания графов, таких как алгоритм Тарьяна и алгоритм Хопкрофта-Тарьяна, которые находят применение в различных областях, включая географическую информационную систему (ГИС) и компьютерную графику.
Триангуляция графов: Триангуляция позволяет разбить граф на треугольники, что упрощает анализ и обработку данных. Она широко используется в графических системах и вычислительной геометрии.