%В основе алгоритма построения маршрутов с заданной дистанцией лежит метод направленного блуждания на графе. В начале алгоритма пользователь задает стартовую и конечную точки маршрута, а также желаемую дистанцию. Начальная погрешность отклонения от желаемой дистанции составляет 10\%, при работе алгоритма погрешность может динамически корректироваться в зависимости от заданной целевой дистанции. Алгоритм инициализируется данными параметрами и начинает процесс построения маршрута.
%
%Ключевой особенностью алгоритма является динамический выбор следующей вершины маршрута, учитывая факторы, такие как расстояния до вершин, приближение к целевой вершине, сонаправленность с вектором направления, повторное посещение вершин. Это позволяет строить маршруты, которые соответствуют заданной дистанции. В данном алгоритме маршруты оптимизируются по 3 ключевым факторам: удаленность достижимых вершин (данный фактор позволяет избежать возврата в предыдущую вершину при наличии альтернативных вершин), штраф за повторное посещение вершин (чем давнее вершина была посещена, тем меньше назначается штраф), штраф за отдаление от целевой вершины (данный фактор позволяет учитывать направление движения и стимулирует выбор вершин, которые приближают к цели).
%
%У описанного метода существует недостаток в необходимости ручного выбора шаблона маршрута, а также его угла. Для решения данного недостатка предлагается многократный запуск метода с различными шаблонами маршрута и шагом угла в 10 градусов. В результате определится список маршрутов, многие из которых будут геометрически слабо различимы, по этой причине необходимо определять кластеры геометрически схожих маршрутов. Определять в каждом из кластеров маршрутов центроиды, максимально удовлетворяющие пользовательским фильтрам.


% |
% |
% |
%\ /
% Нужен рефакторинг

В основе алгоритма построения маршрутов с заданной дистанцией лежит метод направленного блуждания на графе. Этот метод позволяет создавать маршруты, учитывая заданные пользователем параметры, такие как стартовая и конечная точки, а также желаемая дистанция. Начальная погрешность отклонения от желаемой дистанции составляет 10\%, и в процессе работы алгоритма эта погрешность может динамически корректироваться. Алгоритм инициализируется данными параметрами и начинает процесс построения маршрута.

%Инициализация и основные параметры
На начальном этапе пользователь задает начальную и конечную точки маршрута. Алгоритм также получает желаемую дистанцию маршрута и допустимую погрешность. Эти параметры служат исходными данными для инициализации алгоритма. В качестве основы для алгоритма используется граф, где вершины представляют собой значимые точки (например, перекрестки или достопримечательности), а ребра – возможные пути между этими точками.

%Процесс построения маршрута
Ключевой особенностью алгоритма является динамический выбор следующей вершины маршрута. Этот выбор основывается на нескольких факторах:

\begin{itemize}
\item удаленность достижимых вершин: Этот фактор позволяет избежать возврата в предыдущую вершину при наличии альтернативных вершин, обеспечивая более плавный и логичный маршрут;
\item штраф за повторное посещение вершин: Чем давнее вершина была посещена, тем меньше назначается штраф. Это позволяет алгоритму минимизировать повторное прохождение уже пройденных участков;
\item штраф за отдаление от целевой вершины: Этот фактор учитывает направление движения и стимулирует выбор вершин, которые приближают к цели.
\end{itemize}

Эти факторы работают в комплексе, позволяя алгоритму выбирать наиболее подходящие вершины для формирования маршрута, соответствующего заданной дистанции.

%Марковские сети и вероятностный выбор
Алгоритм генерации маршрутов имеет тесную связь с марковскими сетями. Марковские сети используются для моделирования вероятностных процессов, где текущее состояние системы зависит только от предыдущего состояния. В контексте построения маршрутов, каждое состояние представляет собой текущую вершину графа, а переходы между состояниями (вершинами) осуществляются на основе вероятностных правил, учитывающих вышеописанные факторы. Это позволяет создавать маршруты, которые не только соответствуют заданным параметрам, но и оптимизируются по нескольким критериям.

%Многократный запуск и кластеризация маршрутов
У описанного метода существует недостаток в необходимости ручного выбора шаблона маршрута и угла. Для решения этого недостатка предлагается многократный запуск метода с различными шаблонами маршрута и шагом угла в 10 градусов. В результате создается список маршрутов, многие из которых будут геометрически слабо различимы. Для оптимизации этого процесса проводится кластеризация маршрутов.

Кластеризация позволяет группировать геометрически схожие маршруты, определяя в каждом кластере центроиды – маршруты, максимально удовлетворяющие пользовательским фильтрам. Эти центроиды представляют собой наиболее оптимальные маршруты в каждой группе, обеспечивая пользователю наилучшие варианты с учетом заданных параметров и условий.

%Преимущества и выводы
Использование описанного алгоритма позволяет создавать маршруты с фиксированной дистанцией и заданными пользователем фильтрами, обеспечивая высокую степень гибкости и адаптивности. Динамическая корректировка погрешности, вероятностный выбор вершин на основе марковских сетей и кластеризация схожих маршрутов делают этот алгоритм эффективным и удобным для пользователей. Таким образом, веб-сервис предоставляет пользователям уникальные возможности для планирования и оптимизации своих маршрутов, значительно улучшая пользовательский опыт и удовлетворяя разнообразные потребности.





