\usepackage{extsizes}
\usepackage{cmap} % для кодировки шрифтов в pdf
\usepackage[T2A]{fontenc}
\usepackage[utf8]{inputenc}
\usepackage[russian]{babel}
\usepackage{pscyr}

\usepackage{graphicx} % для вставки картинок
\usepackage{amssymb,amsfonts,amsmath,amsthm} % математические дополнения от АМС
\usepackage{indentfirst} % отделять первую строку раздела абзацным отступом тоже
\usepackage[usenames,dvipsnames]{color} % названия цветов
\usepackage{makecell}
\usepackage{multirow} % улучшенное форматирование таблиц
\usepackage{ulem} % подчеркивания

\linespread{1.3} % полуторный интервал
\renewcommand{\rmdefault}{ftm} % Times New Roman
\frenchspacing

\usepackage{tabularx}
\usepackage{longtable}

% нумерация страниц
\usepackage{fancyhdr}
\pagestyle{fancy}
\fancyhf{}
\fancyhead[R]{\thepage}
\fancyheadoffset{0mm}
\fancyfootoffset{0mm}
\setlength{\headheight}{17pt}
\renewcommand{\headrulewidth}{0pt}
\renewcommand{\footrulewidth}{0pt}
\fancypagestyle{plain}{
    \fancyhf{}
    \rhead{\thepage}}
\setcounter{page}{4} % начать нумерацию страниц с определенного номера


% подписи под изображениями и таблицами
\usepackage[tableposition=top]{caption}
\usepackage{subcaption}
\DeclareCaptionLabelFormat{gostfigure}{Рисунок #2}
\DeclareCaptionLabelFormat{gosttable}{Таблица #2}
\DeclareCaptionLabelSeparator{gost}{~---~}
\captionsetup{labelsep=gost}
\captionsetup[figure]{labelformat=gostfigure}
\captionsetup[table]{labelformat=gosttable}
\renewcommand{\thesubfigure}{\asbuk{subfigure}}


% заголовки
\usepackage{titlesec}

\titleformat{\chapter}[display]
{\filcenter}
{\MakeUppercase{\chaptertitlename} \thechapter}
{8pt}
{\bfseries}{}

\titleformat{\section}
{\normalsize\bfseries}
{\thesection}
{1em}{}

\titleformat{\subsection}
{\normalsize\bfseries}
{\thesubsection}
{1em}{}

% Настройка вертикальных и горизонтальных отступов
\titlespacing*{\chapter}{0pt}{-30pt}{8pt}
\titlespacing*{\section}{\parindent}{*4}{*4}
\titlespacing*{\subsection}{\parindent}{*4}{*4}


% поля
\usepackage{geometry}
\geometry{left=3cm}
\geometry{right=1.7cm}
\geometry{top=2.2cm}
\geometry{bottom=2cm}


% списки
\usepackage{enumitem}
\AddEnumerateCounter{\asbuk}{\@asbuk}{м)}
\makeatother
\setlist{nolistsep}
\renewcommand{\labelitemi}{-}
\renewcommand{\labelenumi}{\asbuk{enumi})}
\renewcommand{\labelenumii}{\arabic{enumii})}


% оглавление
\usepackage{tocloft}
\renewcommand{\cfttoctitlefont}{\hspace{0.38\textwidth} \bfseries\MakeUppercase}
\renewcommand{\cftbeforetoctitleskip}{-1em}
\renewcommand{\cftaftertoctitle}{\mbox{}\hfill \\ \mbox{}\hfill{\footnotesize Стр.}\vspace{-2.5em}}
\renewcommand{\cftchapfont}{\normalsize\bfseries \MakeUppercase{\chaptername} }
\renewcommand{\cftsecfont}{\hspace{31pt}}
\renewcommand{\cftsubsecfont}{\hspace{11pt}}
\renewcommand{\cftbeforechapskip}{1em}
\renewcommand{\cftparskip}{-1mm}
\renewcommand{\cftdotsep}{1}
\setcounter{tocdepth}{2} % задать глубину оглавления — до subsection включительно

\newcommand{\empline}{\mbox{}\newline}
\newcommand{\likechapterheading}[1]{
    \begin{center}
        \textbf{\MakeUppercase{#1}}
    \end{center}
    \empline}

\makeatletter
\renewcommand{\@dotsep}{2}
\newcommand{\l@likechapter}[2]{{\bfseries\@dottedtocline{0}{0pt}{0pt}{#1}{#2}}}
\makeatother
\newcommand{\likechapter}[1]{
    \likechapterheading{#1}
    \addcontentsline{toc}{likechapter}{\MakeUppercase{#1}}}


% списки иллюстраций, листингов и таблиц
\renewcommand{\cftlottitlefont}{\hspace{0.38\textwidth} \bfseries}
\renewcommand{\cftbeforelottitleskip}{-1em}
\renewcommand{\cftafterlottitle}{\mbox{}\hfill \\ \mbox{}\hfill{\footnotesize Стр.}\vspace{-2.5em}}

\renewcommand{\cftloftitlefont}{\hspace{0.34\textwidth} \bfseries}
\renewcommand{\cftbeforeloftitleskip}{-1em}
\renewcommand{\cftafterloftitle}{\mbox{}\hfill \\ \mbox{}\hfill{\footnotesize Стр.}\vspace{-2.5em}}

\usepackage{listing}
\renewcommand{\listlistingname}{Список программных листингов}

\addto\captionsrussian{
    \renewcommand{\cftfigfont}{Рисунок }
    \renewcommand{\cfttabfont}{Таблица }
}


% список литературы
\usepackage[square,numbers,sort&compress]{natbib}
\renewcommand{\bibnumfmt}[1]{#1.\hfill} % нумерация источников в самом списке — через точку
\renewcommand{\bibsection}{\likechapter{Список использованных источников}} % заголовок специального раздела
\setlength{\bibsep}{0pt}


% счетчик страниц, рисунков, таблиц, источников
\newcounter{totfigures}
\newcounter{tottables}
\makeatletter
\AtEndDocument{%
    \addtocounter{totfigures}{\value{figure}}%
    \addtocounter{tottables}{\value{table}}%
    \immediate\write\@mainaux{%
        \string\gdef\string\totfig{\number\value{totfigures}}%
        \string\gdef\string\tottab{\number\value{tottables}}%
    }%
}
\makeatother

\usepackage{etoolbox}
\pretocmd{\chapter}{\addtocounter{totfigures}{\value{figure}}}{}{}
\pretocmd{\chapter}{\addtocounter{tottables}{\value{table}}}{}{}

\newcounter{totreferences}
\pretocmd{\bibitem}{\addtocounter{totreferences}{1}}{}{}



\usepackage{subfiles} % для включения подфайлов
\usepackage{xcolor} % Play with colors


\usepackage{caption}
\usepackage{booktabs}

\usepackage{minted} % для вставки кода
\newenvironment{code}{\captionsetup{type=listing}}{}

\usepackage{amssymb}  % готические буквы
\usepackage{wrapfig}  % обтекание картинок текстом

\usepackage{float}


